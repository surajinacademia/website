%!TeX program = pdflatex
% Suraj Kumar Sahu's Curriculum Vitae
% Email: ssahu2@ucmerced.edu
% Web: https://sahusuraj.com/
% Repo: https://github.com/gboeing/cv

\documentclass[11pt,letterpaper]{report}
\usepackage[T1]{fontenc} % output T1 font encoding (8-bit) for accented characters as single glyph
\usepackage[strict,autostyle]{csquotes} % smart and nestable quote marks
\usepackage[USenglish]{babel} % regionalize hyphens, quote marks, etc automatically
\usepackage{microtype}% improve text appearance with kerning, etc
\usepackage{datetime} % enable formatting of date output
\usepackage{tabto}    % make nice tabbing
\usepackage{ragged2e}
\usepackage{hyperref} % enable hyperlinks and pdf metadata
\usepackage{geometry} % manually set page margins
\usepackage{enumitem} % enumerate with [resume] option
\usepackage{titlesec} % allow custom section fonts
\usepackage{setspace} % custom line spacing
\usepackage{fontawesome5}
\usepackage{multicol}
% what is your name?
\newcommand{\myname}{Suraj Kumar Sahu}
% select default typefaces
\usepackage{ebgaramond} % document's serif typeface
\usepackage{helvet}     % document's sans serif typeface
% how far to tab for list items with left-aligned date: different fonts need different widths
\newcommand{\listtabwidth}{2.8cm}
% define font to use as document's title
\newcommand{\namefont}[1]{{\normalfont\bfseries\Huge{#1}}}
% set section heading fonts and before/after spacing
\SetTracking{encoding=*, family=\sfdefault}{30} % increase sans serif headings tracking
\titleformat{\section}{\lsstyle\sffamily\small\bfseries\uppercase}{}{}{}{}
\titlespacing{\section}{0pt}{30pt plus 4pt minus 4pt}{8pt plus 2pt minus 2pt}
% set subsection heading fonts and before/after spacing
\titleformat{\subsection}{\lsstyle\sffamily\footnotesize\bfseries}{}{}{}{}
\titlespacing{\subsection}{0pt}{16pt plus 4pt minus 4pt}{4pt plus 2pt minus 2pt}
% set page margins (assumes letter paper)
\geometry{body={7.0in, 9.5in},
    left=0.75in,
    top=0.75in}
% prevent paragraph indentation
\setlength\parindent{0em}
% set line spacing
\setstretch{0.9}
% define space between list items
\newcommand{\listitemspace}{0.25em}
% make unordered lists without bullets and use compact spacing
\renewenvironment{itemize}
{\begin{list}{}{\setlength{\leftmargin}{0em}
                \setlength{\parskip}{0em}
                \setlength{\itemsep}{\listitemspace}
                \setlength{\parsep}{\listitemspace}}}{\end{list}}
% make tabbed lists so content is left-aligned next to years
\TabPositions{\listtabwidth}
\newlist{tablist}{description}{3}
\setlist[tablist]{leftmargin=\listtabwidth,
    labelindent=0em,
    topsep=0em,
    partopsep=0em,
    itemsep=\listitemspace,
    parsep=\listitemspace,
    font=\normalfont}
% print only the month and year when using \today
\newdateformat{monthyeardate}{\monthname[\THEMONTH] \THEYEAR}
% define hyperlink appearance and metadata for pdf properties
\hypersetup{
    colorlinks  = true,
    urlcolor    = black,
    citecolor   = black,
    linkcolor   = green,
    pdfauthor   = {\myname},
    pdfkeywords = {Biophysics, Collective Dynamics, Physics, Ph.D.},
    pdftitle    = {\myname: Curriculum Vitae},
    pdfsubject  = {Curriculum Vitae},
    pdfpagemode = UseNone
}

\begin{document}
    \raggedright{}
    % display your name as the document title
    \namefont{\myname}
    % affiliation and contact info blocks
    \vspace{1em}
    \begin{minipage}[t]{0.700\textwidth}
        % current primary affiliation, left-aligned
        Department of Physics \\
        School of Natural Sciences \\
        University of California Merced\\
        5200 Lake Rd, Merced, CA-95343
    \end{minipage}
    \begin{minipage}[t]{0.295\textwidth}
        % contact info details, right-aligned
        \flushright{}
        \href{https://www.sahusuraj.com}{website  \faAtlas} \\
        \href{mailto:ssahu2@ucmerced.edu}{ e-mail \faEnvelope} \\ 
        \href{https://www.linkedin.com/in/ucmercedsuraj}{linkedin \faLinkedin} \\
           
            
    \end{minipage}
    %Info about the document
    \begin{center}           
            Updated \monthyeardate\today
    \end{center}

%-----------PERSONAL STATEMENT---------------
\section*{PERSONAL STATEMENTS}
        
    \subsection*{Research Interests}
        \justifying
        \hspace{5mm} I use biophysics theory and computational modeling to study self-organization of multicellular networks. My PhD research focuses on how cell migration, cell-cell adhesion, and cell-ECM interactions lead to self-organization and remodeling of vascular networks. My goal is to develop multi-scale model to investigate the stability, functionality and robustness of such transport networks. My research provides insights into emergent phenomena in living systems from a physics perspective can can aid in understanding various developmental cardiovascular diseases.

        \begin{center}
            Keywords: \textbf{Mechanobiology, Self-Organization, Agent Based Modeling, Vascular Development, Biophysics}
        \end{center}

    \subsection*{Outreach}
        \justifying
        \hspace{5mm} I have experience organizing and designing science outreach activities for diverse audiences. I have helped develop engaging hands-on activities like toys-from-trash, Foldscope, and CellPaint. I have lead and assisted in events/talks for encouraging students to pursue graduate studies. 

        \begin{center}
            Keywords: \textbf{Community outreach, Toys from Trash, Science Storytelling, CellPaint, Foldscope, Science Communication, Popular Science}
        \end{center}

    \subsection*{Teaching}
        \justifying
        \hspace{5mm} I have assisted and partially taught introductory physics courses, including experimental labs and discussions. I am passionate about teaching in ways that brings excitement and engagement to the topics. My effort goes towards motivating students who may struggle in grasping the scientific concepts. I aim to teach courses at the intersection of physics and biology in a holistic and storytelling approach. I wish to develop pedagogy that introduces numerical computation and efficient but safe use of LLMs as learning assistant. 

        \begin{center}
            Keywords: \textbf{Undergraduate Physics, Biophysics, Active learning, Computational Modeling}
        \end{center}

   
\newpage    
    
%-----------ACADEMIC JOURNEY---------------
\section*{ACADEMIC JOURNEY}
    \subsection*{Academic Positions}
        \begin{tablist}
            \item[Jan 21 - present] \tab{\href{http://gopinathanlab.ucmerced.edu/people.html}{\textbf{Graduate Research Assistant}}}  at \href{http://gopinathanlab.ucmerced.edu}{Gopinathan Group}, Department of Physics, University of California Merced.
    
            \item[Jan 21 - present] \tab{\textbf{Graduate Teaching Assistant}} at \href{https://physics.ucmerced.edu}{Department of Physics}, University of California Merced. 

            \item[Aug 17-May 19] \tab{\href{https://sites.google.com/view/biomoldyn/}{\textbf{Graduate Student Researcher}}} at \href{https://sites.google.com/view/biomoldyn/}{Computational Biophysics Group}, Department of Physics and Astronomy, National Institute of Technology Rourkela, Odisha, India.

        \end{tablist}
    
    \subsection*{Education}
        \begin{tablist}
            \item[Jan 21 - present] \tab{\href{https://physics.ucmerced.edu/content/suraj-sahu}{\textbf{PhD Candidate}}} Department of Physics, University of California Merced, California, USA.
    
            \item[Aug 17-May 19] \tab{\textbf{Master of Science in Physics}} Department of Physics and Astronomy, National Institute of Technology Rourkela, Odisha, India.
    
            \item[Jul 14 - Jun 17] \tab{\textbf{Bachelor of Science (Honors in Physics)}} D.R. Nayapalli College, Utkal University, Odisha, India.
        \end{tablist}
    

%-----------SERVICE--------------
\section*{SERVICE}
    \begin{tablist} 
        \item[Aug 23-Aug 25] \tab{\href{https://cemb.upenn.edu/members/professional-development-and-training-opportunities/}{\textbf{Trainee Leadership Council}} at Center for Engineering Mechanobiology (CEMB)}
                \\Organized tutorials, research presentations, professional development workshops and webinars.
        \item[Jun 21-Jun 23] \tab{\textbf{President} at \href{https://sites.google.com/view/biophysics-graduate-club/home}{\textbf{Graduate Biophysics Club}}}
                \\Led science outreach events, facilitated journal club discussions in biophysics research, professional development workshops and networking events.
        \item[Aug 24-Aug 25] \tab{\textbf{\href{https://graduatedivision.ucmerced.edu/PMP}{\textbf{GradExcel Peer Mentor}}}}
                \\Mentored graduate students, supporting personal well-being and professional development.
    \end{tablist}

 
     
%-----------AWARDS--------------
\section*{AWARDS AND FELLOWSHIPS}
        \begin{tablist}
            \item[2025] \tab{Center for Engineering MechanoBiology(CEMB) Summer Research Fellowship, Center for Celluar and Biomolecular Machines(CCBM) Travel Award, Physics graduate group travel fellowship}
            \item[2024] \tab{Physics graduate group travel fellowship, GradExcel Peer Mentor Award} 
            \item[2023] \tab{CCBM Outreach Fellowship, CCBM Travel Fellowship, Physics graduate group travel fellowship}
            \item[2022] \tab{Physics graduate group travel fellowship, Bobcat Summer STEM Academy Fellowship} 
    \end{tablist}


%-----------PUBLICATIONS--------------
\section*{PUBLICATIONS}
    \subsection*{Journal Articles}
        \begin{tablist}    
            \item[Jan, 2021] \tab{}\textbf{Suraj, S.}, M. Biswas, \href{https://doi.org/10.1016/j.jmgm.2021.107936}{" Modeling protein association from homogeneous to mixed environments: A reaction-diffusion dynamics approach."}, \textit{Journal of Molecular Graphics and Modeling}, vol.107, pp.107936.
    \end{tablist}

\newpage
%-----------RESEARCH--------------
\section*{RESEARCH}
    
    \subsection*{Graduate Research Projects}
        \begin{tablist}

            \item[Oct 24-pres] \tab{\textbf{Compaction of ECM by Multicellular Networks}} in collaboration with \textbf{\href{https://sites.ucmerced.edu/kdasbiswas}{Dasbiswas Lab}}\\Computational modeling of compaction and remodeling of collagen matrix due to contractile forces by multicellular networks of fibroblasts cells. \textit{*In preparation}

            \item[Aug 24-pres] \tab{\textbf{Mechanobiology of Cell-Cell Junction Formation and Adhesion stability}} 
            \\Part 1: Cadherin kinetics and actomyosin dynamics in cell junction formation and maturation. \textit{*In preparation}. 
            \\Part 2: Mechanochemical model of feedback loops leading to self-organization of cell-cell junction strength and stability. 

            \item[Jan 21 - pres] \tab{\href{https://meetings.aps.org/Meeting/MAR22/Session/Z03.3}{\textbf{Agent-based modeling of Vasculogenesis}} in collaboration with  \href{https://www.sindilab.com}{\textbf{Sindi Lab}} and \href{http://kara-mccloskey.squarespace.com}{\textbf{Kara E. McCloskey Lab.}}}
            \\Using a Agent based Network dynamics model we study the development of vascular network formation and remodeling. Quantifying the functionality, resilience and adaptability of transport networks  \textit{*In preparation}
            
            \item[Jan 21 - Jun 21] \tab{\textbf{DNA Target-Site Search optimization by DNA binding proteins}} \\We explored the reasons behind how DNA binding proteins find their target sites on a DNA faster than the diffusion limited search strategy. \textit{*Lab research rotation project.}
                 
            \item[Aug 18-Dec 21] \tab{\textbf{\href{https://drive.google.com/file/d/1AjtjmQMciYm_7xYxGh5CERsXeFOR8A0K/view?usp=sharing}{Thermodynamics and Kinetics of Macromolecular Crowding effects on Protein Reaction}}}
            \\We explored how crowder size, composition and nature of interactions affects the kinetics and thermodynamics of a binary protein association by using a coarse-grained reaction diffusion system(ReaDDy).   
          \end{tablist}
          
    
    \subsection*{Other Research Projects}
    
        \begin{tablist}
                        
            \item[Summer 18] \tab{\href{https://drive.google.com/file/d/1AtUgdT2HTfizxT766BorXWUXXJA-PHN6/view?usp=sharing}{\textbf{Dynamics of Indian Languages and Language Competition}} \href{http://gopinathanlab.ucmerced.edu}{in collaboration with \href{https://www.linkedin.com/in/rashi-agarwal-88321014b/?originalSubdomain=in}{\textbf{Rashi Agarwal}}}}
            \\Nearly 90\% of indigenous languages in India are facing direct threat of extinction. Using a non-linear dynamical model we predicted the missing data of certain scheduled languages languages like Kashmiri, Tamil, Dogri and Assamese.

            \item[Fall 18] \tab{\textbf{Steiner Problem in collaboration with Rashi Agarwal}. 
            \\On finding the shortest distance between points on 2D using Soap films.}
                
        \end{tablist} 
        


%-----------CONFERENCES--------------
\section*{CONFERENCES AND WORKSHOPS}

    \subsection*{Organizer/Lead}
        \begin{tablist}
            \item[Spring, 2025] \tab{\textbf{Workshop on AI Tools for Research and Data Analysis}, University of California Merced, Organizer and Instructor}
            
            \item[Summer, 2024] \tab{\href{https://cemb.upenn.edu/programs/graduate-training/boot-camp/}{\textbf{Center for Engineering and Mechanobiology Boot camp}} Project Leader and Instructor, University of Pennsylvania, Philadelphia}
     
        \end{tablist}
    
    \subsection*{Attendee}
        \begin{tablist}    
            \item[Mar 2025] \tab{\href{https://summit.aps.org/events/MAR-B58}{\textbf{2025 March Meeting, American Physical Society}}, Los Angeles, 2025} 
            \\Presentation: Stability of Cell-Cell Junctions: Balancing Cortical Tension and Cadherin Aggregation at cell interface during cell-cell separation. Author: Sahu S, Gopinathan A.

            \item[Dec 2024] \tab{\href{https://www.ascb.org/cellbio2024/}{\textbf{Cell Bio 2024, ASCB | EMBO Meeting}}, San Diego, 2024} 
            \\Poster: Balancing Cortical Tension and Adhesive Force for Stable Cell Junctions. Author: Sahu S, Gopinathan A.
            
            \item[Mar 2024] \tab{\href{https://meetings.aps.org/Meeting/MAR24/Session/Z27.3}{\textbf{2024 March Meeting, American Physical Society}}, Minneapolis, 2024} \\Presentation: Modeling the mechanics of cell-cell junction formation and dynamics in vascular networks. Author: Sahu S, Gopinathan A.
        
            \item[Mar 2023] \tab{\href{https://meetings.aps.org/Meeting/MAR23/Session/T06.13}{\textbf{2023 March Meeting, American Physical Society}}, Las Vegas, 2023} 
            \\Presentation: Particle-Based Simulation of the Assembly and Mechanical Remodeling of Vascular Network. Author: Sahu S, Gopinathan A. Sindi S. McCloskey K.  Kuhn M., Zamora J. 
            
            \item[Mar 2023] \tab{\href{https://convention.nsbe.org}{\textbf{National Society of Black Engineers (NSBE)}}}
            \\ Exhibition: Research opportunities in the \href{https://cemb.upenn.edu}{\textbf{Center for Engineering Mechanobiology (CEMB)}} 

            \item[May 2023] \tab{NSF site visit at \href{https://cemb.upenn.edu}{\textbf{Center for Engineering Mechanobiology (CEMB), University of Pennsylvania, Philadelphia}}} 
            \\Poster: Overview of research projects at Gopinathan Group, University of California Merced.
        
            \item[2022] \tab{\href{https://meetings.aps.org/Meeting/MAR22/Session/Z03.3}{\textbf{2022 March Meeting, American Physical Society}}, Chicago, 2022} 
            \\Presentation: Agent Based Simulation of Vasculogenesis. Author: Sahu S, Gopinathan A. McCloskey K.  Kuhn M., Zamora J. 
            
            \item[Feb 2022] \tab{\href{https://emerging-researchers.org}{\textbf{Emerging Researcher National Conference}}}
            \\Exhibition: Research opportunities in the \href{https://cemb.upenn.edu}{\textbf{Center for Engineering Mechanobiology (CEMB)}}.
    
            \item[July 2022] \tab{\href{https://cemb.upenn.edu/programs/graduate-training/boot-camp/}{\textbf{Center for Engineering and Mechanobiology Boot-camp}}, Washington University St. Louis}
 
        \end{tablist}
    
    \subsection*{Outreach Activities}
        \begin{tablist}
            \item[Aug 24-Aug 25] \tab{\href{https://cemb.upenn.edu/members/professional-development-and-training-opportunities/}{\textbf{Research in Motion (RiM) Series}}}
                \\The CEMB Trainee Leadership Council hosts a monthly series Research in Motion (RiM) for trainees to share their research.
        
            \item[Sept 2024] \tab{Speaker at \textbf{Bahujan Scholars Network}}
                \\Graduate program application series: Guidance on applying for graduate schools.
        
            \item[Jan 2024] \tab{Presenter at \href{https://www.digitalnalanda.com}{\textbf{Digital Nalanda}}}
                \\Microscopic Marvels: Exploring the Tiny Wonders of the Living World using Foldscope. 
                Led by Suraj Sahu and Disha Kuzhively.
        
            \item[Aug 2023] \tab{Presenter at \textbf{Center for Engineering Mechanobiology (CEMB)}}
                \\Demonstrated tools for science outreach for mechanobiology pedagogy to High School teachers.
        
            \item[July 2023] \tab{Organizer at \href{https://sites.google.com/view/biophysics-graduate-club/home}{\textbf{Science of Coronavirus}}}
                \\Hosted a science outreach event for schools using CellPaint to illustrate the science of coronavirus. co-led by Joey McMertien

            \item[Aug 2022] \tab{Planning Committee at \textbf{The Franklin Institute}, Philadelphia}
                \\Contributed to designing mobile museum exhibit on mechanobiology.

            \item[June 2022] \tab{Presenter at \textbf{Center for Cellular and Biomolecular Machines (CCBM)}}
            \\Explored microorganisms using the Foldscope. Led by Jocelyn Ochoa, Anuvetha Govindranjan, Bhavya Mishra.
            
            \item[July 2022] \tab{Instructor at \href{https://calteach.ucmerced.edu/bobcat-summer-stem-academy}{\textbf{Bobcat Summer STEM Academy}}}
            \\Taught electrical circuits to middle school students through interactive demonstrations.
        
            \item[Fall 2021] \tab{Volunteer at \href{https://merced-ca.aauw.net/science-camps/}{\textbf{Mother/Daughter Science Camp}}}
                \\Served with the American Association of University Women (AAUW), led by Dr. Petia Gueorguieva.
        
            \item[June 2021] \tab{Presenter at \href{https://calteach.ucmerced.edu/sites/calteach.ucmerced.edu/files/documents/2021_ccbm_virtual_sessions_1.pdf}{\textbf{The Science of Flocks and Swarms}}}
            \\Demonstrated the physics of flocking and ant foraging using NetLogo, alongside Prof. Ajay Gopinathan, Suraj Sahu, and Ritwika VPS.
        
        \end{tablist}


%-----------GRADUATE COURSEWORK--------------
\section*{GRADUATE COURSEWORK}
    \begin{tablist}
        \item[Physics] \tab{Classical Mechanics, Electrodynamics, Statistical Mechanics, Quantum Mechanics, Non-linear Dynamics and Chaos, Condensed Matter Theory, Atomic and Molecular Physics}
      
        \item[Life Science] \tab{Cell and Cellular Techniques, Basics in Molecular Medicine, Recombinant DNA Technology, Basic Biophysics }
      
        \item[Comp Sci] \tab{Computational Physics, Classical Molecular Simulation, Numerical Mathematical Methods for Physics, Machine Learning $\&$ Statistics for Physics and Astronomy.}    
    \end{tablist}
 

%-----------SKILLS---------------
\section*{SKILLS}   
    \subsection*{Research Skills}
        \begin{tablist}
            \item[Modeling] \tab{Agent-Based Modeling, Particle-Based Simulations, Reaction-Diffusion Systems, Stochastic Modeling, Network Dynamics, Biophysics Theory}
            \item[Computation] \tab{AI Augmented Research Workflow, High-Performance Computing, Prompt Engineering for Coding and Data Analysis}
        \end{tablist}

    \subsection*{Softwares and Tools}
        \begin{tablist}
            \item[Languages] \tab{Python(NumPy, SciPy, Matplotlib, Pandas, NetworkX), Julia, LaTeX}
            \item[Softwares] \tab{ReaDDy (Molecular Dynamics), Cursor AI, NetLogo}
        \end{tablist}



%-----------REFERENCES---------------
\section*{REFERENCES}
    \begin{multicols}{3}
    \flushleft
    \href{https://www.ucmerced.edu/content/ajay-gopinathan}{\textbf{Prof. Ajay Gopinathan}} \href{mailto:agopinathan@ucmerced.edu}{\faEnvelope}\\ 
    Department of Physics, CCBM\\
    School of Natural Sciences\\
    University of California Merced\\
    
    \vfill\null
    
    \columnbreak
    \flushleft
    \href{https://naturalsciences.ucmerced.edu/people/suzanne-sindi}{\textbf{Prof. Suzanne Sindi}} \href{mailto:ssindi@ucmerced.edu}{\faEnvelope}\\
    Department of Applied Mathematics\\
    School of Natural Sciences\\
    University of California Merced\\
    
    \vfill\null
    
    \columnbreak
    \flushleft
    \href{https://ccbm.ucmerced.edu/content/kinjal-dasbiswas}{\textbf{Asst. Prof. Kinjal Dasbiswas}} \href{mailto:kdasbiswas@ucmerced.edu}{\faEnvelope}\\
    Department of Physics, CCBM,\\
    School of Natural Sciences\\
    University of California Merced\\
    
    \end{multicols}
    


\vfill
 


\begin{center}
     \textit{"Books! And cleverness! There are more important things! — Friendship! And Bravery!"}
 \end{center}
    \flushright{- Hermione Granger(Harry Potter and the Philosopher's Stone)}


\end{document}